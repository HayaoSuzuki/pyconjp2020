\documentclass[dvipdfmx,12pt,notheorems]{beamer}
%%%% 和文用 %%%%%
\usepackage{bxdpx-beamer}
\usepackage{pxjahyper}
\usepackage{minijs}%和文用
\renewcommand{\kanjifamilydefault}{\gtdefault}%和文用

%%%% スライドの見た目 %%%%%
\usetheme{Madrid}
\usefonttheme{professionalfonts}
\setbeamertemplate{frametitle}[default][center]
\setbeamertemplate{navigation symbols}{}
\setbeamercovered{transparent}%好みに応じてどうぞ)
\setbeamertemplate{blocks}[rounded]
\useinnertheme{circles}
\setbeamertemplate{footline}[page number]
\setbeamerfont{footline}{size=\normalsize,series=\bfseries}
\setbeamercolor{footline}{fg=black,bg=black}
%%%%

%%%% 定義環境 %%%%%
\usepackage{amsmath,amssymb}
\usepackage{amsthm}
\theoremstyle{definition}
\newtheorem{theorem}{定理}
\newtheorem{definition}{定義}
\newtheorem{proposition}{命題}
\newtheorem{lemma}{補題}
\newtheorem{corollary}{系}
\newtheorem{conjecture}{予想}
\newtheorem*{remark}{Remark}
\renewcommand{\proofname}{}
%%%%%%%%%

%%%%% フォント基本設定 %%%%%
\usepackage[T1]{fontenc}%8bit フォント
\usepackage{textcomp}%欧文フォントの追加
\usepackage[utf8]{inputenc}%文字コードをUTF-8
\usepackage[deluxe]{otf}%otfパッケージ
\usepackage{lxfonts}%数式・英文ローマン体を Lxfont にする
\usepackage{bm}%数式太字
%%%%%%%%%%

%%%%% PythonTeX %%%%%
\usepackage[makestderr]{pythontex}
\restartpythontexsession{\thesection}
 
\title{How to Use In-Memory Streams}
\author[Hayao]{Hayao Suzuki}
\institute[PyCon JP 2020]{PyCon JP 2020}
\date{August 28, 2020}

\begin{document}

\begin{frame}[plain]\frametitle{}
\titlepage %表紙
\end{frame}

\section{Self Introduction}

\begin{frame}\frametitle{Who am I ?}

\begin{block}{お前誰よ}
\begin{description}
\item[Name] Hayao Suzuki(鈴木 駿)
\item[Twitter] \href{https://twitter.com/CardinalXaro}{@CardinalXaro}
\item[Work] Python Programmer at iRidge, Inc.
\end{description}
\end{block}

\end{frame}

\begin{frame}\frametitle{Who am I ?}

\begin{block}{Reviewer of Technical Books}
\begin{itemize}
\item \structure{Effective Python 第2版}(O'Reilly Japan)
\item \structure{動かして学ぶ量子コンピュータプログラミング}(O'Reilly Japan)
\end{itemize}
\end{block}

\begin{block}{Talks}
\begin{itemize}
\item \structure{Symbolic Mathematics using SymPy}(PyCon JP 2018)
\item \structure{Elementary Number Theory with Python}(PyCon mini Hiroshima 2019)
\item \structure{Do you know cmath module?}(PyCon mini Shizuoka)
\end{itemize}
\end{block}
Lists are here  \url{https://xaro.hatenablog.jp/}
\end{frame}

\section{built-in function: open}

\begin{frame}\frametitle{Today's Theme}
\begin{center}
\huge{In-Memory Streams}
\end{center}
\end{frame}

\subsection{open is useful}

\subsection{open with Disk IO}

\section{In-Memory Streams}

\subsection{StringIO}

\subsection{BytesIO}

\section{Case Study}

\subsection{ZIP Compression in Memory using StringIO and BytesIO}

\section{Conclusion}



\end{document}
